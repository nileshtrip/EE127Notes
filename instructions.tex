\documentclass[12pt]{article}

\usepackage{macros}

\begin{document}

\title{Scribe Instructions for EE127}
\date{}
\maketitle

{\bf\color{red} Please read carefully.}

\begin{itemize}
\item Each lecture will be scribed by 2--3 students who will receive extra credit for their efforts.

\item Scribes should use the sample notes (contained in the Lecture0 folder) in the Github (\url{https://github.com/nileshtrip/EE127Notes}) as a Latex template. If you are not familiar with Latex consider this an opportunity to learn\footnote{You might find this tutorial helpful if you are unfamiliar: \url{https://www.overleaf.com/latex/learn/free-online-introduction-to-latex-part-1} .}! Latex is an essential tool for those working in science and engineering.

\item Scribes should produce a single \emph{Lecturex.zip} (where $x$ denotes the lecture number) file in \textit{exactly} the format of the Lecture0 folder. This will be due $1$ week after the lecture date and should be emailed to \emph{nilesh\_tripuraneni@berkeley.edu}.

\item Students should strive to produce \textit{high-quality notes}\footnote{This is why we are allowing several students to scribe a single lecture!}, verify correctness of the material, and produce illustrative figures to the best of their ability for the content where helpful. You might it useful if all group members take hand-written notes during lecture and later meet to conglomerate and proof-read them.

\item Students should base their content on their notes from the lecture and available supplemental resources.

\item Figures can be created by either by scanning in clean, hand-written diagrams, creating electronic figures, or pulling from available source material (including lecture slides/course texts).

\item Scribing sign-ups should hopefully stabilize during the first 2 weeks of the course (the link will be available on Piazza under the scribing tab). After this initial sign-up, I will freeze a version of the sign-ups sheet about two weeks after the start of the course. At this point if you cannot scribe a certain lecture it is your responsibility to find a person (or group to switch with).


\item Scribes are required to provide references in bibtex format when referring to any external material. See sample notes (contained in the Lecture0 folder) for examples of how to cite references from the \emph{refs.bib}. If you need to add an additional reference, Google scholar is good way to Bibtex blurbs to the \emph{refs.bib}.
\item Pull requests with typos and other suggested changes are welcome; please direct comments to the Piazza.
\end{itemize}

\subsection*{List of common macros and useful stylistic markers}

\begin{itemize}
\item Real numbers $\R$, use \latexcommand{R}
\item Real-valued functions, use letters $f, g, h$
\item Domain $\Omega\subseteq\R^n$ of a function if not all of $\R^n$, use \latexcommand{domain}
\item Scalars, use greek letters
\item Vectors, use letters $u, v, w$
\item Matrices, use capital letters $A, B, \dots$
\item For transpose sign~$\trans$, use \latexcommand{trans}, e.g., $A^\trans$
\item Inner products, use \latexcommand{langle} and \latexcommand{rangle}, or use transposes.
\item For code, use the \href{https://en.wikibooks.org/wiki/LaTeX/Source_Code_Listings}{listings} package.
\item See {\tt macros.sty} for other available macros.

\end{itemize}

\subsection{Some requirements}

\begin{itemize}
\item Your section should start with a brief summary of what it contains.
\item Write full sentences that don't start with symbols.
\item Use standard latex conventions.
\item Don't use ``||`` for norms, use ``\textbackslash |''.
\end{itemize}

\end{document}
